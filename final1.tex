\documentclass[a4paper,11pt]{article}
\usepackage[utf8]{inputenc}
\usepackage[russian]{babel}
\usepackage[T1]{fontenc}
\usepackage{amssymb,amsmath,graphicx,indentfirst}
\usepackage{caption}
\usepackage{color}
\usepackage{listings}
\usepackage{qtree}
\usepackage[unicode]{hyperref}

\setlength{\parskip}{1ex plus 0.5ex minus 0.2ex}
\captionsetup[figure]{labelformat=empty}
\captionsetup[figure]{justification=centering}
\lstset{keywordstyle=\color{blue}\bfseries, basicstyle=\footnotesize}
\lstset{breaklines=true, breakatwhitespace=true}
\lstset{extendedchars=false, language=Caml, defaultdialect=[Objective]Caml}

\newcommand{\lam}{\ensuremath{\lambda}}
\newcommand{\term}[1]{\textbf{#1}}
\renewcommand{\emph}[1]{\textit{#1}}
\newcommand{\itemhard}{\stepcounter{enumi}\item[\theenumi $^\ast$.]}
\newcommand{\warpar}{\par\medskip}
\newcommand{\bydef}{\ensuremath{\stackrel{\text{\upshape df}}{=}}}
\newcommand{\Natural}{\mathbb N}
\newcommand{\Real}{\mathbb R}
\newcommand{\Complex}{\mathbb C}
\newcommand{\prob}[1]{\mathbb P \left\{#1\right\}}
\DeclareMathOperator*{\argmin}{\arg\!\min}
\DeclareMathOperator*{\argmax}{\arg\!\max}
\newcommand{\<}{\langle}
\renewcommand{\>}{\rangle}

\newcommand{\startmbout}{\color{BrickRed}}
\newcommand{\finishmbout}{\color{black}\upshape}
\newenvironment{mbout}{\startmbout}{\finishmbout}
\newenvironment{xcode}{\begin{quote}\ttfamily\footnotesize}{\end{quote}\par}
\newenvironment{xcomment}{\par\begin{longtable}{ \mid p{0.95\textwidth}}}{\end{longtable}\par}

\lstnewenvironment{cs}
        {\lstset{language={[Sharp]C}}}
        {}
\lstnewenvironment{fsharp}
        {\lstset{language={[Objective]Caml},
                    morecomment=[l]//}}
        {}
\lstnewenvironment{haskell}
        {\lstset{language=Haskell}}
        {}

\author{Олег Смирнов, Александр Полозов \\
\texttt{oleg.smirnov@gmail.com}, \texttt{polozov.alex@gmail.com}}
\date{16 декабря 2011 г.}
\title{Введение в функциональное программирование -- Экзамен. Первый семестр}
\begin{document}
\maketitle
\newpage

\section*{Билет \No 1}
\begin{enumerate}
\item \textbf{Функции высших порядков. Замыкания (2 балла).}\\
  Пусть имеется глобальный словарь \textit{cache: Dictionary<'a, 'b>},
  и существуют две функции работы с ним:
  \textit{tryGetValue: Dictionary<'a, 'b> $\to$ 'a $\to$ bool * 'b}, достающая
  значение по ключу (если таковой существует в словаре, иначе возвращает
  false в первом элементе кортежа) и
  \textit{addValue: Dictionary<'a, 'b> $\to$ 'a $\to$ 'b $\to$ unit},
  добавляющая пару <<ключ-значение>> в словарь. Реализуйте оператор
  мемоизации \textit{memo: ('a $\to$ 'b) $\to$ ('a $\to$ 'b)}, возвращающий
  мемоизированную версию поданной на вход функции. Мемоизированная функция
  сохраняет результаты всех своих вычислений и не вычисляет их повторно,
  если результат можно достать из кэша.

\item \textbf{Списочные комбинаторы (3 балла).} \\
  Реализуйте функцию 
  $zip3~:~'\!a~list~\rightarrow~'\!b~list~\rightarrow~'\!c~list~\rightarrow
  ('\!a*'\!b*'\!c)~list$,
  которая превращает три списка одинаковой длины в один список кортежей-троек:
  \begin{lstlisting}
    zip3 [1; -1; 0] [2; -2; 0] [3; -3; 0];;
    [(1, 2, 3); (-1, -2, -3); (0, 0, 0)]
  \end{lstlisting}
  \emph{Примечание:} если можно, реализуйте сразу итеративно.

\item \textbf{Списочные гомоморфизмы и технология MapReduce (6 баллов).}\\
  Пусть задан граф ссылок между страницами сети Интернет. На вход MapReduce
  подаются данные в следующем формате:
  \begin{lstlisting}[language={}, basicstyle=\ttfamily\scriptsize]
    A    B
    A    C
    B    C
    B    D
    C    B
    C    A
    C    B
  \end{lstlisting}
  Необходимо построить список обратных ссылок, т.е для каждой страницы вывести,
  какие страницы ссылаются \emph{на неё}. Например:
  \begin{lstlisting}[language={}, basicstyle=\ttfamily\scriptsize]
    A    C
    B    A, C
    C    A, B
    D    B
  \end{lstlisting}
  Напишите решение, используя шаблоны функций $mapper$ и $reducer$:
  \begin{lstlisting}
    let mapper (key: string) (value: string): seq<string * string> =
    // TODO
    let reducer (key: string) (values: seq<string>): (string * seq<string>) =
    // TODO
  \end{lstlisting}
\end{enumerate}
\newpage

\section*{Билет \No 2}
\begin{enumerate}
\item \textbf{Рекурсия и итерация. Хвостовая рекурсия (2 балла)}\\
  Напишите итеративную версию бинарного поиска в отсортированном массиве.
  Функция должна принимать четыре аргумента:
  \begin{itemize}
  \item искомый элемент
  \item минимальный индекс массива
  \item максимальный индекс массива
  \item функцию-предикат, которая проверяет, находится ли искомый элемент по
    заданному индексу: $isFound~:~int~\rightarrow~int~\rightarrow~bool$
  \end{itemize}

\item \textbf{Списочные комбинаторы (3 балла).}\\
  Реализуйте функцию $mapi~:~(int~\rightarrow~'\!a~\rightarrow
  '\!b)~\rightarrow~'\!a~list~\rightarrow~'\!b~list$, которая преобразует один
  список в другой список с помощью указанной функции. В качестве первого
  аргумента функции передаётся номер текущего обрабатываемого элемента:
  \begin{lstlisting}
    mapi (fun i x -> (i, x)) [1; 2; 3];;
    [(0, 1); (1, 2); (2, 3)]
  \end{lstlisting}
  \emph{Примечание:} если можно, реализуйте сразу итеративно.

\item \textbf{Сканирующие пробеги, сегментированные пробеги (6 баллов).}\\
  Задана рекуррентная последовательность:
  $$\left\{
  \begin{aligned}
    x_0 &= 1 \\
    x_i &= i^2 x_{i-1} + C_{100}^i
  \end{aligned}
  \right.$$
  Укажите все операции, константы и покажите необходимые условия, необходимые,
  чтобы вычислить значения этой последовательности методом сканирующих пробегов.
\end{enumerate}
\newpage

\section*{Билет \No 3}
\begin{enumerate}
\item \textbf{Рекурсия и итерация. Хвостовая рекурсия (2 балла)}\\
  Напишите итеративную версию бинарного алгоритма Евклида для поиска
  наибольшего общего делителя (GCD) двух чисел:
  \begin{enumerate}
  \item $gcd(0, n) = n$; $gcd(m, 0) = m$; $gcd(m, m) = m$
  \item Если $m$, $n$ чётные, то $gcd(m, n) = 2*gcd(m/2, n/2)$
  \item Если $m$ чётное, $n$ нечётное, то $gcd(m, n) = gcd(m/2, n)$
  \item Если $n$ чётное, $m$ нечётное, то $gcd(m, n) = gcd(m, n/2)$
  \item Если $m$, $n$ нечётные и $n > m$, то $gcd(m, n) = gcd((n - m)/2, m)$
  \item Если $m$, $n$ нечётные и $n < m$, то $gcd(m, n) = gcd((m - n)/2, n)$
  \end{enumerate}

\item \textbf{Списочные комбинаторы (3 балла).} \\
  Реализуйте функцию
  $exists2~:~('\!a~\rightarrow~'\!b~\rightarrow~bool)~\rightarrow~'\!a~list~
  \rightarrow~'\!b~list~\rightarrow~bool$,
  которая принимает функцию-предикат от двух аргументов, два списка элементов
  и проверяет, удовлетворяет ли любая пара из элементов данному предикату:
  \begin{lstlisting}
    exists2 (fun a b -> a = b) [1; 2; 3] [3; 2; 1];;
    true
    exists2 (fun a b -> a = b) [1; 1; 1] [0; 0; 0];;
    false
  \end{lstlisting}
  \emph{Примечание:} если можно, реализуйте сразу итеративно.

\item \textbf{Списочные гомоморфизмы и технология MapReduce (6 баллов).}\\
  Задан документ, состоящий из строк. Каждая строка состоит из множества
  слов. Необходимо построить обратный индекс, т.е. для каждого слова вывести
  список строк, в котором оно встречается. Напишите решение, используя шаблоны
  функций $mapper$ и $reducer$:
  \begin{lstlisting}
    let mapper (key: string) (value: string): seq<string * string> =
    // TODO
    let reducer (key: string) (values: seq<string>): (string * seq<string>) =
    // TODO
  \end{lstlisting}
\end{enumerate}
\newpage

\section*{Билет \No 4}
\begin{enumerate}
\item \textbf{Функции высших порядков. Замыкания (2 балла).}\\
  Имея действительную функцию $f$, определим \textit{сглаженную} функцию
  $f$ как $x \mapsto \frac{f(x-dx) + f(x) + f(x+dx)}{3}$. Напишите оператор
  $n$-кратного сглаживания:
  $$ smoothN: (float \to float) \to int \to (float \to float) $$

\item \textbf{Алгебраические типы данных (3 балла).}\\
  Пусть задан следующий тип данных, задающий древовидную структуру файловой
  системы:
  \begin{fsharp}
    type Node =
    | File of string
    | Directory of string * Node list
  \end{fsharp}
  Выведите на консоль структуру дерева файловой системы с именами всех файлов
  и папок, по образцу:
  \begin{lstlisting}[language={}, basicstyle=\ttfamily\scriptsize]
|-dir1
| |-dir3
|   |-file1
|   |-dir4
|     |-file5
|     |-file6
|     |-dir5
|       |-file7
|       |-dir6
|       | |-file8
|       | |-dir8
|       |   |-file11
|       |   |-file12
|       |   |-file13
|       |   |-dir11
|       |     |-dir12
|       |       |-file16
|       |-dir7
|         |-file9
|         |-file10
|         |-dir9
|         | |-file14
|         |-dir10
|           |-file15
|-dir2
  |-file2
  \end{lstlisting}

\item \textbf{Сканирующие пробеги, сегментированные пробеги (6 баллов).}\\
  Используя сегментированные пробеги и стандартные операции на их основе,
  реализуйте параллельный нерекурсивный алгоритм QuickSort (как и на лекции,
  рекурсия заменяется на итерации параллельной работы с несколькими сегментами).
\end{enumerate}
\newpage

\section*{Билет \No 5}
\begin{enumerate}
\item \textbf{Функции высших порядков. Замыкания (3 балла).}\\
  Допустим, F\# позволяет присваивать значения переменным с помощью
  Pascal-синтаксиса (\texttt{x := 5}). Используя замыкания, реализуйте
  функции $makeWeighter: (int \to bool) \to \;?$. $makeWeighter$ возвращает
  некий <<объект>>, который потом можно использовать с двумя другими
  функциями: $addValue: \;? \to int \to unit$ и $getAverage: \;? \to double$.
  Семантика поведения такова: при создании объекта в $makeWeighter$
  передается предикат $p$. Функция $addValue$ добавляет в <<базу>> объекта
  новое переданное числовое значение-аргумент, если оно удовлетворяет этому
  предикату. Функция $getAverage$ позволяет в любой момент времени получить
  среднее элементов в базе. Пример работы:
  \begin{fsharp}
    let weighter = makeWeighter (fun x -> x % 2 == 0)
    addValue weighter 2
    addValue weighter 13
    addValue weighter 6
    addValue weighter 8
    let x = getAverage weighter // 5.33 = (2 + 6 + 8) / 3
  \end{fsharp}

\item \textbf{Списочные комбинаторы (3 балла).}\\
  Реализуйте функцию $unzip~:~('\!a~*~'\!b)~list~\rightarrow~
  '\!a~list~*~'\!b~list$, которая принимает список пар элементов и возвращает
  пару из двух списков:
  \begin{lstlisting}
    unzip [(1,2); (3,4)];;
    ([1; 3], [2; 4])
  \end{lstlisting}
  \emph{Примечание:} если можно, реализуйте сразу итеративно.

\item \textbf{Моноидальные вычисления в деревьях (6 баллов).}\\
  Имеется массив данных типа int: [1; 7; 10; 2; 14; 13; 5; 5; 1; 3;
  4; 42; 11; 2; 2; 9]. Постройте на них Rope, размер chunk = 2, мера
  $f(x) = [x\text{ является простым числом}]$, моноидальная операция~--~XOR.
  Укажите все аннотации.
\end{enumerate}
\newpage 

\section*{Билет \No 6}
\begin{enumerate}
\item \textbf{Функции высших порядков. Замыкания (2 балла).}\\
  Напишите процедуру $makeFracFunction$ построения вычислителя цепной дроби:
  $$f(x) = \cfrac{N_1(x)}{D_1(x) + \cfrac{N_2(x)}{D_2(x) + \cdots + \cfrac{N_k(x)}{D_k(x)}}}
  $$
  Сигнатура: $makeFracFunction: (int \to float \to float) \to
  (int \to float \to float) \to (float \to float)$. Она принимает на вход
  функции $N(i, x) = N_i(x)$ и $D(i, x) = D_i(x)$.
\item \textbf{Алгебраические типы данных (3 балла).}\\
  Пусть задан следующий алгебраический тип данных, описывающий
  арифметические выражения:
  \begin{lstlisting}
    type Expr = 
    | Plus of Expr*Expr
    | Minus of Expr*Expr
    | Var of string
    | Const of int
  \end{lstlisting}
  Напишите функцию $prettyPrint~:~Expr~\rightarrow~string$, которая приводит 
  выражение к ``читабельному'' виду. Функция должна расставлять скобки по мере
  необходимости. Пример работы:
  \begin{lstlisting}
    prettyPrint (Plus (Const 5) (Minus (Const 7) (Const 8)));;
    "(3+(7-8))"
  \end{lstlisting}

\item \textbf{Обобщение свёртки на АТД. Катаморфизмы (6 баллов).}\\
  Пусть задан алгебраический тип данных $Prop$, описывающий выражения логики
  высказываний. Тип данных включает поддержку констант, переменных, а также
  операций конъюнкции, дизъюнкции и отрицания.
  \begin{enumerate}
  \item Напишите тип данных для ``контекста свёртки''. Контекстом называется
    такой тип данных, где рекурсивные вхождения исходного типа ($Prop$ в данном
    примере) заменены на некоторый параметр $'\!a$ -- вычисленные значения
    предыдущих свёрток:
    \begin{lstlisting}
      type 'a propContext =
      | ...
    \end{lstlisting}
  \item Напишите нерекурсивную функцию-вычислитель для выражения с 
    использованием контекста: $eval~:~bool~propContext~\rightarrow~bool$.
  \end{enumerate}
\end{enumerate}
\newpage 

\section*{Билет \No 7}
\begin{enumerate}
\item \textbf{Списочные комбинаторы (3 балла).}\\
  Реализуйте функцию $count~:~('\!a~\rightarrow~bool)~
  \rightarrow~'\!a~list~\rightarrow~int$, которая принимает функцию-предикат 
  и список элементов, и возвращает число элементов, удовлетворяющих данному
  предикату:
  \begin{lstlisting}
    count (fun x -> x > 0) [-1; 0; 1; 1; 1];;
    3
  \end{lstlisting}
  \emph{Примечание:} если можно, реализуйте сразу итеративно.

\item \textbf{Алгебраические типы данных (3 балла).}\\
  Пусть задан следующий алгебраический тип данных, описывающий выражения 
  логики высказываний:
  \begin{lstlisting}
    type Prop =
    | True
    | Not of Prop
    | And of Prop*Prop
    | Or of Prop*Prop
    | Var of string
  \end{lstlisting}
  Напишите функцию $prettyPrint~:~Prop~\rightarrow~string$, которая приводит 
  выражение к ``читабельному'' виду. Функция должна расставлять скобки по мере
  необходимости. Пример работы:
  \begin{lstlisting}
    prettyPrint (Or (Var(``B''), Not(Var(``B''))));;
    "(B||~B)"
    prettyPrint (And (True, True));;
    "(True&&True)"
  \end{lstlisting}

\item \textbf{Моноидальные вычисления в деревьях (6 баллов).}\\
  Имеется Rope на данных типа char, длины chunk`ов неравномерны.
  Восемь chunk`ов -- ``Наша'', ``Таня'', ``громко'', ``плачет'',
  ``уронила'', ``в'', ``речку'', ``мячик''. Тип данных аннотаций --
  числа (целые, вещественные или булевые на ваше усмотрение). Укажите
  такую функцию меры $f(x)$, моноидальную операцию $\circ$ и предикат $p$,
  чтобы разрезание по этому предикату произошло на пятом chunk`е; на первом
  chunk`е; вообще не произошло (предикат везде false).
\end{enumerate}
\newpage 

\section*{Билет \No 8}
\begin{enumerate}
\item \textbf{Рекурсия и итерация. Хвостовая рекурсия (2 балла)}\\
  Реализуйте рекурсивную функцию перевода системы счисления 
  $convertBase: int \to int \to int\;list$. Задано число в десятичной системе
  счисления и основание целевой системы; получить список цифр числа в целевой
  системе.

\item \textbf{Алгебраические типы данных (3 балла).}\\
  Пусть задан следующий алгебраический тип данных, описывающий
  арифметические выражения:
  \begin{lstlisting}
    type Expr = 
    | Plus of Expr*Expr
    | Times of Expr*Expr
    | Const of int
  \end{lstlisting}
  Напишите функцию $polishPrint~:~Expr~\rightarrow~string$, которая выводит
  выражение в обратной польской нотации. Т.е. в таком виде, когда операнды
  расположены перед знаками операций. Пример работы:
  \begin{lstlisting}
    polishPrint (Plus(Times(Plus(Const(1), Const(2)), Const(4)), Const(3)));;
    "1 2 + 4 * 3 +"
  \end{lstlisting}

\item \textbf{Обобщение свёртки на АТД. Катаморфизмы (6 баллов).}\\
  Пусть задан алгебраический тип данных, описывающий произвольный 
  XML-документ в виде $N$-арного дерева. Тип данных включает поддержку 
  элементов (тэгов), атрибутов и содержимого для тэгов. Пример документа:
  \begin{lstlisting}[language=XML]
    <item>
      <child attr="1">Hi!</child>
      <new_item>
        <child attr="2" another_attr="text"/>
      </new_item>
    </item>
  \end{lstlisting}
  \begin{enumerate}
  \item Напишите тип данных для ``контекста свёртки''. Контекстом называется
    такой тип данных, где рекурсивные вхождения исходного типа заменены на
    некоторый параметр $'\!a$ -- вычисленные значения предыдущих свёрток:
    \begin{lstlisting}
      type 'a xmlContext =
      | ...
    \end{lstlisting}
  \item Напишите нерекурсивную функцию, которая преобразует XML-документ
    в текстовый вид, удобный для чтения: $prettyPrint~:~string~xmlContext 
    \rightarrow string$
    (см. пример выше).
  \end{enumerate}
\end{enumerate}

\end{document}
