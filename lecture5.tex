\documentclass[a4paper,11pt]{article}
\usepackage[utf8]{inputenc}
\usepackage[russian]{babel}
\usepackage[T1]{fontenc}
\usepackage{amssymb,amsmath,graphicx,indentfirst}
\usepackage{caption}
\usepackage{color}
\usepackage{listings}
\usepackage[unicode]{hyperref}
\usepackage{qtree}

\setlength{\parskip}{1ex plus 0.5ex minus 0.2ex}
\captionsetup[figure]{labelformat=empty}
\captionsetup[figure]{justification=centering}
\lstset{keywordstyle=\color{blue}\bfseries}
\lstset{extendedchars=false, language=Caml, defaultdialect=[Objective]Caml}

\author{Олег Смирнов\\
\texttt{oleg.smirnov@gmail.com}}
\date{14 октября 2011 г.}
\title{Введение в функциональное программирование -- Лекция 5. Алгебраические
типы данных. Сопоставление с образцом}

\begin{document}
\maketitle
\tableofcontents
\newpage

\section*{Цель лекции}
\begin{itemize}
\item Алгебраические типы данных и сопоставление с образцом
\item Полиморфные типы данных
\item Списки, деревья, тип $option$
\end{itemize}

\section{Алгебраические типы данных}
Пусть необходимо написать программу, которая бы оперировала арифметическим
выражениями вида
\begin{equation}
  2*x + x*x + 7
  \label{eq:1}
\end{equation}
Такие выражения удобно представлять в виде синтаксических деревьев, где операции
задаются узлами, их операнды -- потомками, а константы хранятся в листях дерева:

\Tree [.$+$ [.$+$ [.$*$ $x$ $x$ ] $7$ ] [.$*$ $x$ $2$ ] ]

В наших выражениях листья могут быть двух типов: константы, заданые целым 
числом, и переменные, заданые строкой-именем переменной. Определим новый тип 
данных, удобный для представления арифметических выражений. Пусть переменные 
задаются с префиксом $Var$, константы с префиксом $Const$, а сумма и умножение
-- с префиксами $Sum$ и $Mul$ соответственно:

\Tree [.$Sum$ [.$Sum$ [.$Mul$ $Var~x$ $Var~x$ ] $Const~7$ ]%
[.$Mul$ $Var~x$ $Const~2$ ] ]

\begin{lstlisting}
type Expr = 
  | Var of string 
  | Const of int
  | Sum of Expr * Expr
  | Mul of Expr * Expr
\end{lstlisting}

В такой нотации вертикальная черта обозначает выбор из нескольких вариантов, а
знак умножения в строчках с $Sum$ и $Mul$ -- просто пару из двух значений типа
$Expr$. После такого определения можно записать выражение типа $Expr$:
\begin{lstlisting}
  let constSeven = Const(5)
  let sumOneTwo = Sum(Const(1), Const(2))
\end{lstlisting}
Выражение \ref{eq:1} можно определить как:
\begin{lstlisting}
  Sum(Sum(Mul(Var("x"), Var("x")), Const(7)),
    Mul(Var("x"), Const(2)))
\end{lstlisting}

\section{Сопоставление с образцом}
Тип данных $Expr$ определён как набор вариантов. Для работы с такой структурой
удобно использовать оператор $match$, который сопоставляет аргумент с различными
образцами. Напишем функцию, которая будет вычислять значение выражения:
\begin{lstlisting}
  let rec eval Expr =
    match Expr with
    | Const(c) -> c
    | Var(v) -> getVar v
    | Sum(e1, e2) -> eval e1 + eval e2
    | Mul(e1, e2) -> eval e1 + eval e2
\end{lstlisting}
В левой части каждой строки после оператора $match$ находится \emph{образец}. В
правой части после стрелочки -- действие, которое выполнится, если аргумент
$match$ совпадёт с данным образцом. В образцах можно использовать связывание
переменных. Например, $e1$ и $e2$ будут связаны с аргументами $Mul$ и $Sum$.

Расширим тип данных $Expr$, чтобы он позволял задавать не только сложение и
умножение, но и другие бинарные операции. Каждая операция будет задана тремя
аргументами: сигнатурой, первым аргументом и вторым аргументов. Аналогично
определим унарные операции:
\begin{lstlisting}
  ...
  | Binary of Op2 * Expr * Expr
  | Unary of Op1 * Expr
\end{lstlisting}
Здесь $Op2$ и $Op1$ -- сигнатуры бинарных и унарных операций соответственно. Их
можно определить в качестве самостоятельных типов данных:
\begin{lstlisting}
  Op2 = Sum | Sub | Mul | Div
  Op1 = Neg
\end{lstlisting}
Теперь можно расширить функцию $eval$ для поддержки новых операций:
\begin{lstlisting}
  let rec eval Expr =
    match Expr with
    ...
    | Binary(op, e1, e2) -> 
            match op with 
            | Sum -> eval e1 + eval e2
            | Mul -> eval e1 * eval e2
            | Div -> eval e1 / eval e2
            | Sub -> eval e1 - eval e2
    | Unary(op, e1) -> 
            match op with 
            | Neg -> not eval e1
            | _ -> eval e1
\end{lstlisting}
Допустим, на момент реализации известна только операция отрицания $Neg$.
Последняя строчка функции -- это ``заглушка'', которая сработает для любого
неизвестного значения $op$. Здесь нижнее подчёркивание означает ``что угодно'' --
универсальный образец, подходящий для всех значений. Важно помнить, что правила
$match$ выполняются \emph{сверху вниз}. Т.е. все правила, записанные после
сопоставления с $\_$, никогда не выполнятся.

Предположим, что функция $eval$ должна работать только с положительными
константами, а все отрицательные -- ``сбрасывать'' в ноль. Это можно записать
следующим правилом:
\begin{lstlisting}
  match Expr with
  | Const(c) when c < 0 -> 0
  | Const(c) -> c
\end{lstlisting}
Условие после ключевого слова $when$ называется ``стражником'' от англ. guard.

Правила сопоставления с образцом можно комбинировать, делая код более
читабельным. Функцию вычисления факториала можно записать так:
\begin{lstlisting}
  let rec fact n =
    match n with
    | 0 -> 1
    | 1 -> 1
    | n -> n * fact (n - 1)
\end{lstlisting}
А можно короче, объединив в два условия с одинаковым действием:
\begin{lstlisting}
  let rec fact n =
    match n with
    | 0 | 1 -> 1
    | n -> n * fact (n - 1)
\end{lstlisting}

\section{Абстракция данных}
В такой нотации, как в примерах выше, $Const$, $Var$ и т.п. называют 
\emph{конструкторами}, т.к. они позволяют создать экземпляр данного типа. Кроме
конструкторов ещё определяют \emph{селекторы} и \emph{предикаты}. Например,
для типа $Expr$ удобно написать предикат $isConst~:~Expr~\rightarrow~bool$ и
$isVar~:~Expr~\rightarrow~bool$:
\begin{lstlisting}
  let isConst e =
    match e with
    | Const(_) -> true
    | _ -> false

  let isVar e =
    match e with
    | Var(_) -> true
    | _ -> false
\end{lstlisting}

Примером селектора может быть функция $getConst$, которая возвращает значение,
если её аргумент -- константа.

Для работы с графами можно задать тип данных $Graph$, у которого конструкторами
будут $Node$ и $Link$ -- вершина и ребро графа, соответственно. А предикатом
может быть функция $isConnected$, проверяющая, является ли граф связным.

Тип данных, заданный набором своих конструкторов, селекторов и предикатов,
называют \emph{абстрактным типом данных}, т.к. его внутреннее представление
скрыто от программиста. Граф типа $Graph$ может храниться в виде матрицы
смежности, матрицы инцидентности или в виде списка вершин. Его \emph{интерфейс}
из конструкторов, селекторов и предикатов для любой реализации останется 
неизменным.

Тип данных $Expr$ называется \emph{алгебраическим типом данных}, т.к. он
представлен в виде размеченного объединения декартовых произведений множеств
или, другими словами, в виде \emph{суммы произведений типов}. Действительно,
вертикальная черта, объединяющая варианты выбора между $Var$, $Const$ и др.,
эквивалентна логической операции ``или'' или суммированию. Знак умножения
в аргументах конструкторов означает декартово произведение и задаёт кортеж из
соответствующих типов. А $Const$, $Var$ и т.п. -- это \emph{метки} для таких 
кортежей.
\section{Логика высказываний}
Рассмотрим ещё один пример. Выражения \emph{логики высказываний} можно 
определить с помощью константы $T$ (``истина''), и операторов отрицания,
конъюнкции и дизъюнкции:

\begin{lstlisting}
type proposition = 
  | True
  | Not of proposition
  | And of proposition * proposition
  | Or of proposition * proposition
\end{lstlisting}

Несколько примеров выражений, записанных с помощью такого типа данных:
\begin{enumerate}
\item $\lnot T \lor \lnot \lnot T$
  \begin{lstlisting}
    let prop1 = Or(Not(True), Not(Not(True)))
  \end{lstlisting}
\item $\lnot \lnot \lnot \lnot T$
  \begin{lstlisting}
    let prop2 = Not(Not(Not(Not(True))))
  \end{lstlisting}
\item $\lnot (T \land \lnot T) \lor T$
  \begin{lstlisting}
    let prop3 = Or(Not(And(True, Not(True))), True)
  \end{lstlisting}
\end{enumerate}

Вычислить значение выражения можно с помощью рекурсивной функции, аналогично
арифметическому выражению:
\begin{lstlisting}
  let rec eval p =
    match p with
    | True -> true
    | Not(prop) -> not(eval(prop))
    | And(prop1, prop2) -> eval(prop1) && eval(prop2)
    | Or(prop1, prop2) -> eval(prop1) || eval(prop2)
\end{lstlisting}

Функцию $eval$ легко модифицировать так, чтобы она вычисляла операторы в
нормальном порядке. Например, если результат вычисления первого аргумента
оператора $And$ равен $false$, то второй аргумент вычислять уже не нужно.

Добавим в тип данных тернарный оператор $IfThenElse$:
\begin{lstlisting}
type proposition = 
  | True
  | Not of proposition
  | And of proposition * proposition
  | Or of proposition * proposition
  | IfThenElse of proposition * proposition * proposition
\end{lstlisting}

Результат вычисления $IfThenElse$ определяется следующим образом: если
результат вычисления первого аргумента истинен, то вычислить второй
аргумент. Иначе вычислить третий аргумент.
\begin{lstlisting}
    ...
    | IfThenElse(p1, p2, p3) -> 
        if eval(p1) then eval(p2) else eval(p3)
\end{lstlisting}

Теперь наша функция позволяет вычислять условные выражения:
\begin{lstlisting}
  eval(IfThenElse(Not(True), 
       Or(True, Not(True)),
       And(True, Not(True))))
  val it : bool = false
\end{lstlisting}

В тип $proposition$ легко добавить поддержку переменных, получив калькулятор
для логики нулевого порядка.

\section{Деревья и полиморфные типы данных}
Существует несколько способов описания деревьев с помощью алгебраических
типов данных. Разные способы удобны для разных задач. Для простоты пока
будем рассматривать только бинарные деревья из целых чисел.

Способ первый, когда данные находятся в листях дерева:

\Tree [.$\circ$ [.$\circ$ [.$\circ$ $E$ $1$ ]  [.$\circ$ $2$ $3$ ] ] %
[.$\circ$ [.$\circ$ $3$ $4$ ]  [.$\circ$ $5$ $6$ ] ] ]

\begin{lstlisting}
  type Tree1 =
    | Empty
    | Leaf of int
    | Node of Tree1 * Tree1
\end{lstlisting}

Здесь $Empty$ и $Leaf$ -- конструкторы для пустого листа и листа с числом, а
$Node$ -- конструктор узла дерева с двумя потомками.

Способ второй, когда данные находятся в узлах дерева:

\Tree [.$3$ [.$4$ [.$5$ $E$ $E$ ]  [.$8$ $E$ $E$ ] ] %
[.$2$ [.$7$ $E$ $E$ ]  [.$9$ $E$ $E$ ] ] ]

\begin{lstlisting}
  type Tree2 =
    | Empty
    | Node of int * Tree2 * Tree2
\end{lstlisting}

Здесь конструктор $Node$ имеет три аргумента -- значение в узле и два
потомка.

В качестве примера рассмотрим работу с деревьями первого типа. Можно написать
функцию, которая будет умножать на два все значения в листьях дерева.
Результатом работы функции будет дерево того же типа:
\begin{lstlisting}
  let rec mul2 Tree =
    match Tree with 
      | Empty -> Empty
      | Leaf(x) -> Leaf(x*2)
      | Node(t1, t2) -> Node(mul2 t1, mul2 t2)
\end{lstlisting}

Функцию можно обобщить, чтобы она применяла произвольную операцию к значениям
в листьях. Фактически, это будет аналог оператора проекции $map$ для списков:
\begin{lstlisting}
  let rec map op Tree =
    match Tree with 
      | Empty -> Empty
      | Leaf(x) -> Leaf(op x)
      | Node(t1, t2) -> Node(map op t1, map op t2)
\end{lstlisting}

Функции $eval$ для арифметических выражений и для выражений логики высказываний,
по сути, являются аналогами оператора свёртки для списков.

Деревья только для целых чисел не очень полезны. F\# позволяет обобщить этот
тип данных, чтобы в листьях можно было хранить данные произвольного (но только
одного!) типа. В этом случае говорят, что тип $Tree$ \emph{параметризован}
типом $'a$ (читается как $\alpha$):
\begin{lstlisting}
  type 'a Tree =
    | Empty
    | Leaf of 'a
    | Node of 'a Tree * 'a Tree
\end{lstlisting}
Код функции $map$ \emph{не изменится}. Изменится только её тип: 
$map: ('a \rightarrow 'b)~\rightarrow~'a~Tree~\rightarrow~'b~Tree$. Сравните
его с типом функции $map$ для списков.

Можно определить тип, параметризованный более, чем одним подтипом. Синтаксис
F\# в этом случае будет похож на работу с обобщёнными типами в .NET. Например,
тип $Rope$, чередующий значения двух разных типов, можно было бы определить как
$Rope<'a,~'b>$.

Параметризованные типы ещё называют \emph{полиморфными типами данных}.
\section{Списки как алгебраический тип данных}
Списки в F\# -- это полиморфный тип данных, который определён следующим образом:
\begin{lstlisting}
  type 'a list = 
    | Nil
    | Cons of 'a * 'a list
\end{lstlisting}

Где $Cons$ -- это уже известная операция конкатенации $::$, а $Nil$ -- это 
пустой список $[]$. Таким образом, запись $[1; 2; 3]$ -- это просто 
\emph{синтаксический сахар} для эквивалентной записи $1 :: 2 :: 3 :: []$ или
\begin{lstlisting}
  Cons(1, Cons(2, Cons(3, Nil))) 
\end{lstlisting}

Благодаря этому при сопоставлении с образцом можно строить сложные конструкции,
например, ``список, начинающийся на [1; 2]'':
\begin{lstlisting}
  match list with
  | 1 :: 2 :: tail -> tail
\end{lstlisting}
Или ``список, где первый элемент единица, а третий -- тройка'':
\begin{lstlisting}
  match list with
  | 1 :: _ :: 3 -> true
\end{lstlisting}
Если в списке интересует элемент между $1$ и $3$, его можно получить, связав
переменную:
\begin{lstlisting}
  match list with
  | 1 :: n :: 3 -> n
\end{lstlisting}

Такую форму сопоставления можно использовать для любых алгебраических типов
данных. Например, в примере с логикой высказываний можно добавить правило, 
снимающее двойное отрицание перед вычислением выражения:
\begin{lstlisting}
  | Not(Not(e)) -> eval e
  | Not(e) -> not eval e
\end{lstlisting}

Стандартный модуль F\# $List$ содержит ряд полезных селекторов и предикатов
для списков. Например, предикат $List.isEmpty$ и селекторы $List.length$,
$List.head$ и $List.tail$.

\section{Тип необязательных значений}
Предположим, что нам необходимо написать функцию деления двух значений типа
$float$. 
\begin{lstlisting}
  let div x y = x / y
\end{lstlisting}
Как определить поведение функции, если делитель -- ноль? Существует
несколько вариантов:
\begin{enumerate}
\item Вернуть специальное значение типа $float$, зарезервированное для кода
  ошибки. Например, $infinity$ -- бесконечность.
\item Вызвать исключение, ``развернув'' стек вызовов функций.
\item Сгенерировать системное прерывание.
\item Вызвать функцию обработки ошибки, переданную в $div$ в качестве аргумента.
\end{enumerate}

Каждый из подходов имеет собственные недостатки. Например, в первом случае
не всегда бывает возможно зарезервировать одно из значений в качестве
константы -- кода ошибки. Выход -- определить новый тип данных, который будет
хранить или ответ типа $float$, или ошибку:
\begin{lstlisting}
  type floatOrError =
  | Result of float
  | Error
\end{lstlisting}
Очевидно, что такое определение может быть полезно и для других типов. Например,
``$string$ или ошибка'', ``$int$ или ошибка'' и т.д. Эта идея воплощена во
встроенном в F\# типе $option$:
\begin{lstlisting}
  type 'a option =
  | Some of 'a
  | None
\end{lstlisting}
С помощью него можно переписать функцию деления $float$:
\begin{lstlisting}
  let div x y =
    match y with 
    | 0 -> None
    | _ -> Some (x / y)
\end{lstlisting}
Её тип будет $float \rightarrow float \rightarrow float~option$, а результат --
$None$, если была попытка деления на ноль, иначе $Some$ с результатом вычисления.

Для работы с $option$ удобно определить селекторы $isNone$ и $isSome$.

Рассмотрим функцию, которая умножает на два свой аргумент, заданный в виде типа
$int~option$. Она должна проверять два варианта:
\begin{lstlisting}
  let mulTwo arg =
    match arg with
    | None -> None
    | Some(x) -> Some(x*2)
\end{lstlisting}
Можно определить функцию, которая будет применять к аргументу произвольную 
операцию $op$:
\begin{lstlisting}
  let map op arg =
    match arg with
    | None -> None
    | Some(x) -> Some(op x)
\end{lstlisting}

Тип полученной функции: $('a \rightarrow 'b) \rightarrow 'a~option \rightarrow 
'b~option$. Такое определение абсолютно аналогично определению функций $map$ для 
типов $list$ и $Tree$, описанных выше. Во всех трёх случаях функция $map$
принимает в качестве второго аргумента тип, параметризованный типом $'a$ и
``превращает'' его в тип, параметризованный другим типом -- $'b$ с помощью
функции из $'a$ в $'b$, указанной в первом аргументе.

При работе с данными, ``завёрнутыми'' в тип $option$, придётся либо постоянно
проверять два варианта, либо в какой-то момент ``достать'' данные, по
необходимости обработав ошибку. 

Типы данных $list$, $Tree$ и $option$ можно рассматривать как \emph{конструкторы 
типов} или \emph{операторы типов высших порядков}, т.к. они принимают типы в
качестве аргументов и возвращают новые типы в качестве результата. Типы высших
порядков будут подробнее рассмотрены в следующих лекциях.
\nocite{*}
\bibliographystyle{alpha}
\bibliography{lecture5}
\end{document}
