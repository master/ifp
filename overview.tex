\documentclass[a4paper,11pt]{article}
\usepackage[utf8]{inputenc}
\usepackage[russian]{babel}

\author{Олег Смирнов \\
\texttt{oleg.smirnov@gmail.com}}
\title{Введение в функциональное программирование -- аннотация курса}
\date{\today}

\begin{document}
\maketitle
В последнее время функциональное программирование из предмета 
академических исследований превратилось в инструмент для эффективного
решения промышленных задач. Задача курса~--- познакомить слушателей
c основными идиомами функционального подхода и примерами их
применения на практике. Будут рассмотрены верификация программ,
абстракция данных и параллельное программирование в стиле MapReduce.
Рабочие языки курса~--- F\# и Haskell.

Программа основан на курсах ``15-150: Functional Programming'' 
Университета Карнеги--Меллон и ``CS61A: Structure and Interpretation
of Computer Programs'' Университета Беркли.

\subsection*{Цели курса}
\begin{itemize}
\item Дать студентам представление о теоретических основах и практической
реализации современных функциональных языков программирования
\item Познакомить студентов с методом функциональной декомпозиции задачи
и с характерными для функционального стиля приёмами программирования.
\item Рассмотреть наиболее важные приёмы из мира функционального
программирования: рекурсивные и итеративные процессы, функции высших порядков
и замыкания, абстрактные типы данных, доменно-специфичные языки, модель
окружений и бесконечные потоки данных, а также дать введение в системы типов
\end{itemize}

В результате изучения курса студенты приобретают практические навыки
решения задач с помощью функциональных языков программирования.

\subsection*{Литература курса}
\begin{itemize}
\item Абельсон Х., Сассман Дж. ``Структура и интерпретация компьютерных программ''
\item Харрисон Дж. ``Введение в функциональное программирование''
\item Харрисон П., Филд А. ``Функциональное программирование''
\item Пейтон-Джонс С. ``Язык и библиотеки Haskell 98''
\item Пирс Б. ``Типы в языках программирования''
\end{itemize}
\end{document}
