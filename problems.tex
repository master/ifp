\documentclass[a4paper,11pt]{article}
\usepackage[utf8]{inputenc}
\usepackage[russian]{babel}
\usepackage[T1]{fontenc}
\usepackage{amssymb,amsmath,graphicx}
\usepackage{caption}
\usepackage{color}
\usepackage{listings}
\usepackage[unicode]{hyperref}

\setlength{\parskip}{1ex plus 0.5ex minus 0.2ex}
\captionsetup[figure]{labelformat=empty}
\captionsetup[figure]{justification=centering}
\lstset{keywordstyle=\color{blue}\bfseries}
\lstset{extendedchars=false, language=Caml, defaultdialect=[Objective]Caml}

\author{Олег Смирнов, Александр Полозов \\
\texttt{oleg.smirnov@gmail.com}, \texttt{polozov.alex@gmail.com}}
\date{\today}
\title{Введение в функциональное программирование -- задачи}

\begin{document}
\section*{Модуль 1. Темы 1.1-2}
\textbf{Упражнение 1-1} Напишите программу, которая выяснит, какой порядок
применения функций -– аппликативный или нормальный –- используется в F\#? То же
для любого императивного языка на ваш выбор. Подумайте, для каких задач какой
порядок применения будет предпочтительнее.

\end{document}
