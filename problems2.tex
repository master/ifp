\documentclass[a4paper,11pt]{article}
\usepackage[utf8]{inputenc}
\usepackage[russian]{babel}
\usepackage[T1]{fontenc}
\usepackage{amssymb,amsmath,graphicx,indentfirst}
\usepackage{caption}
\usepackage{color}
\usepackage{listings}
\usepackage{enumerate}
\usepackage[unicode]{hyperref}

\setlength{\parskip}{1ex plus 0.5ex minus 0.2ex}
\captionsetup[figure]{labelformat=empty}
\captionsetup[figure]{justification=centering}
\lstset{keywordstyle=\color{blue}\bfseries, basicstyle=\footnotesize}
\lstset{breaklines=true, breakatwhitespace=true}
\lstset{extendedchars=false, language=Caml, defaultdialect=[Objective]Caml}

\author{Олег Смирнов, Александр Полозов \\
\texttt{oleg.smirnov@gmail.com}, \texttt{polozov.alex@gmail.com}}
\date{\today}
\title{Введение в функциональное программирование -- задачи}

\begin{document}
\section*{Модуль 3. Темы 3.1-3.4}
\begin{enumerate}[{1-}1]

\item Реализуйте катаморфизм на синтаксическом дереве, который возвращает
  множество переменных, использованных в выражении.
\item Разработайте алгебраический тип и свёртки для следующих задач.

  \begin{enumerate}[(a)]
  \item У нас есть дерево документа, к примеру к лекции 3.1.
  Элементами дерева могут быть:

  \begin{itemize}
  \item параграф
  \item картинка
  \item ссылка
  \item текст
  \end{itemize}
  В каждом параграфе может быть произвольное количество картинок,
  подпараграфов, текста, ссылок и т.д.

  Задача состоит в том, чтобы извлечь все ссылки для того, чтобы вывести
  их в конце презентации. Напишите для этого катаморфизм.

  \item Напишите катаморфизм, который удалит все картинки из документа.
  \end{enumerate}
\end{enumerate}
\end{document}
