\documentclass[a4paper,11pt]{article}
\usepackage[utf8]{inputenc}
\usepackage[russian]{babel}
\usepackage{hyperref}

\author{Олег Смирнов, Александр Полозов \\
\texttt{oleg.smirnov@gmail.com}, \texttt{polozov.alex@gmail.com}}
\date{\today}
\title{Введение в функциональное программирование -- программа курса}

\begin{document}
\maketitle
\section*{I семестр}
\subsection*{Модуль 1. Основы функционального программирования}
\begin{itemize}
\item Тема 1.1. Вводная лекция. Функциональный подход к программированию.
Применения ФП: верификация, парсинг, параллелизация, ``безошибочность''
\item Тема 1.2. Рекурсивные и итеративные процессы, их отличие и особенности.
Вычисление с аккумулятором. Хвостовая рекурсия. Процесс декомпозиции задачи.
Язык программирования F\#
\end{itemize}
\subsection*{Модуль 2. Практика функционального программирования}
\begin{itemize}
\item Тема 2.1. Функция как объект манипулирования. Функции высших порядков.
Абстракция вычисления. Часто используемые операторы над функциями. 
Каррирование и разрезы
\item Тема 2.2. Замыкания. Проблемы реализации замыканий и их эмуляция в 
императивных языках. Работа со списками, списочные комбинаторы. Пример: LINQ
\item Тема 2.3. Абстракция данных. Алгебраические типы данных. Сопоставление с
образцом. Список как алгебраический тип данных. Деревья и операции над ними.
Пример: примитивный синтаксический разбор
\end{itemize}
\subsection*{Модуль 3. Параллельное программирование}
\begin{itemize}
\item Тема 3.1. Списочная свёртка и её применения. Свёртка деревьев. Обобщение
свёртки на алгебраические типы данных и понятие катаморфизма
\item Тема 3.2. Моноиды. Сбалансированные деревья и дерево отрезков.
Моноидальные вычисления в деревьях
\item Тема 3.3. Сканирующие пробеги и их применения в параллельном
программировании. Сегментированные пробеги. Примеры: алгоритмы параллельного
программирования, использующие идиому пробега –- выпуклая оболочка, максимальный
поток в графе, численные методы
\item Тема 3.4. Списочные гомоморфизмы. Третья теорема о гомоморфизмах. Пример:
технология MapReduce и её открытая реализация Apache Hadoop. Применения
MapReduce: индекс цитирования Web, байесовские классификаторы, пути в графе
\end{itemize}
\section*{II семестр}
\subsection*{Модуль 4. Сочетание различных подходов}
\begin{itemize}
\item Тема 4.1. Концепция изменяемого состояния. Проблемы, связанные с
присваиванием и способы избавления от них в практической разработке
\item Тема 4.2. Чисто функциональные структуры данных. Понятие перситентности.
Семейство структур данных Zipper
\item Тема 4.3. Понятие потока и его связь со списками. Ленивые вычисления. Язык
программирования Haskell
\item Тема 4.4. Продолжения. Основы Continuation Passing Style. Применения:
реализация исключений и сопрограмм. Трансформация рекурсии в цикл
\item Тема 4.5. Проблема побочных эффектов. Монада как абстракция
последовательного вычисления. Примеры монад: List, Maybe, IO.
F\# computational workflows. Асинхронные вычисления
\item Тема 4.6. Монады State и Parser. Пример использование библиотек
Parsec/FParsec для реализации DSL; профессиональный синтаксический разбор
\item Тема 4.7. Классы типов. Моноиды, монады, монады со сложением (MonadPlus),
функторы, аппликативные функторы, ``сворачиваемые'' данные (Foldable)
\end{itemize}
\subsection*{Модуль 5. Дополнительные темы}
\begin{itemize}
\item Тема 5.1. Математические основы функционального программирования.
Бестиповое лямбда-исчисление. Порядок редукций и теорема Черча-Россера.
Теорема о неподвижной точке. Y-комбинатор. Применение: мемоизация
\item Тема 5.2. Введение в теорию типов. Обзор систем типов и алгоритмов
вывода типов, особенности типизированного лямбда-исчисления
\end{itemize}
\end{document}
